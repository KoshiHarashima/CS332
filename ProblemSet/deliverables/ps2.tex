\documentclass[11pt]{article}
\usepackage[margin=.75in]{geometry}
\usepackage[all]{xy}

\usepackage{amsmath,amsthm,amssymb,color,latexsym}
\usepackage{mathtools}
\usepackage{geometry}
\usepackage{booktabs}
\geometry{letterpaper}    
\usepackage{graphicx}
\usepackage{tabto}
\usepackage{setspace}
\usepackage[dvipsnames]{xcolor}

\usepackage[noframe]{showframe}
\usepackage{framed}

\renewenvironment{shaded}{%
  \def\FrameCommand{\fboxsep=\FrameSep \colorbox{shadecolor}}%
  \MakeFramed{\advance\hsize-\width \FrameRestore\FrameRestore}}%
 {\endMakeFramed}
\definecolor{shadecolor}{gray}{0.90}

\newtheorem{problem}{Problem}

\newenvironment{solution}[1][\it{Solution:}]{\textbf{#1 } }

\onehalfspacing

\begin{document}
\begin{titlepage}
   \begin{center}
       \vspace*{9cm}

       \textbf{CS 332 Fall 2025}

       \vspace{0.5cm}
        Problem Set \#2
        \vfill

       \textbf{Koshi Harashima}\\
       Due Date: 10/8, 2025
            
   \end{center}
\end{titlepage}

\pagebreak

%%%%%%%%%%%%%%%%%%%%%%%%%%%%%%%%%%%%%%%%%%%%%%%%%%%%%%%%%%%
%%%%%%%%%%%%%%%%%%%%%%%% Problems %%%%%%%%%%%%%%%%%%%%%%%%%
%%%%%%%%%%%%%%%%%%%%%%%%%%%%%%%%%%%%%%%%%%%%%%%%%%%%%%%%%%%

%--------------------------------------------------------
\begin{shaded}
\section{Problem 1 --- The Doubling Trick}

\textbf{Conclusion:} The per-round regret is at most 
\[
c\,h\sqrt{\frac{\log k}{n}}
\]
for a universal constant \( c \) (e.g., \( c = 8 \)).

\textbf{Reason / steps:}
\begin{enumerate}
    \item Split time into epochs of lengths \( 1,2,4,\dots \). In epoch \( e \) with length \( L_e \), run EW tuned for horizon \( L_e \).
    \item The EW regret per epoch is \( \le 2h\sqrt{L_e\log k} \).
    \item Using sub-additivity of $\max$, total regret is bounded by 
    \[
    2h\sqrt{\log k}\sum_e \sqrt{L_e}.
    \]
    \item Since \( \sum_e \sqrt{L_e} \le (2+\sqrt{2})\sqrt{n} \), total regret 
    \[
    \le 2(2+\sqrt{2})h\sqrt{n\log k}.
    \]
    \item Thus per-round regret 
    \[
    \le 2(2+\sqrt{2})h\sqrt{\frac{\log k}{n}} < 7h\sqrt{\frac{\log k}{n}}.
    \]
\end{enumerate}

\textbf{Note:} Any constant \( c \ge 7 \) works; \( c=10 \) is safe.
\end{shaded}

\pagebreak

\begin{shaded}
\section{Problem 2 --- MAB with \(b\)-element Sets}

\textbf{Conclusion:} A simple uniform-probe $\varepsilon$-greedy algorithm achieves regret
\[
R_n \le c\,h\,b\,(k\log k)^{1/3}n^{2/3}
\]
for a universal constant \( c \) (e.g., \( c = 8 \)).

\textbf{Algorithm:}
\begin{itemize}
    \item Fix exploration rate \( \rho \in (0,1) \).
    \item Each round:
    \begin{enumerate}
        \item With probability \( \rho \), explore: choose one probe arm uniformly at random; fill remaining \( b-1 \) slots arbitrarily.
        \item With probability \( 1-\rho \), exploit: play top-\( b \) arms by empirical means \( \hat{\mu}_j \).
    \end{enumerate}
\end{itemize}

\textbf{Analysis:}
\begin{enumerate}
    \item Exploration regret \( \le \rho n h b \).
    \item Each arm is probed about \( \rho n / k \) times. With Hoeffding + union bound:
    \[
    \max_j |\hat{\mu}_j - \mu_j| \lesssim h\sqrt{\frac{k\log k}{\rho n}}.
    \]
    \item Exploitation regret \( \lesssim 2h b n\sqrt{\frac{k\log k}{\rho n}}. \)
    \item Total regret:
    \[
    R(\rho) \lesssim h b \Big( \rho n + 2n\sqrt{\frac{k\log k}{\rho n}} \Big).
    \]
    \item We have the approximate regret expression:
\[
R(\rho) \approx h b \left( \rho n + n\sqrt{\frac{k\log k}{\rho n}} \right).
\]
Ignoring the constant factors $h,b,n$, define
\[
f(\rho) = \rho + \sqrt{\frac{k\log k}{\rho n}}.
\]
We take the derivative with respect to $\rho$:
\[
f'(\rho) = 1 - \frac{1}{2}\sqrt{\frac{k\log k}{n}}\,\rho^{-3/2}.
\]
The first order condition $f'(\rho)=0$ gives
\[
1 = \frac{1}{2}\sqrt{\frac{k\log k}{n}}\,\rho^{-3/2}.
\]
Solving for $\rho$:
\[
\rho^{3/2} = \frac{1}{2}\sqrt{\frac{k\log k}{n}}
\quad\Rightarrow\quad
\rho = \left(\frac{k\log k}{n}\right)^{1/3} \times \text{(constant)}.
\]
Thus, up to constants,
\[
\rho = c (k\log k / n)^{1/3}.
\]
    
    \item Minimize at \( \rho =  c(k\log k/n)^{1/3} \) giving
    \[
    R_n \le c\,h\,b\,(k\log k)^{1/3}n^{2/3}.
    \]
\end{enumerate}

\textbf{Remarks:}
Using full semi-bandit feedback during exploration can slightly improve the dependence on \( b \), but the simple version already meets the required bound.

\end{shaded}

\end{document}


