\documentclass[11pt]{article}
\usepackage[margin=.75in]{geometry}
\usepackage[all]{xy}

\usepackage{amsmath,amsthm,amssymb,color,latexsym}
\usepackage{mathtools}
\usepackage{geometry}
\usepackage{booktabs}
\geometry{letterpaper}    
\usepackage{graphicx}
\usepackage{tabto}
\usepackage{setspace}
\usepackage[dvipsnames]{xcolor}

\usepackage[noframe]{showframe}
\usepackage{framed}

\renewenvironment{shaded}{%
  \def\FrameCommand{\fboxsep=\FrameSep \colorbox{shadecolor}}%
  \MakeFramed{\advance\hsize-\width \FrameRestore\FrameRestore}}%
 {\endMakeFramed}
\definecolor{shadecolor}{gray}{0.90}

\newtheorem{problem}{Problem}

\newenvironment{solution}[1][\it{Solution:}]{\textbf{#1 } }

\onehalfspacing

\begin{document}
\begin{titlepage}
   \begin{center}
       \vspace*{9cm}

       \textbf{CS 332 Fall 2025}

       \vspace{0.5cm}
        Problem Set \#5
        \vfill

       \textbf{Koshi Harashima}\\
       Due Date: 11/8, 2025
            
   \end{center}
\end{titlepage}

\pagebreak

%%%%%%%%%%%%%%%%%%%%%%%%%%%%%%%%%%%%%%%%%%%%%%%%%%%%%%%%%%%
%%%%%%%%%%%%%%%%%%%%%%%% Problems %%%%%%%%%%%%%%%%%%%%%%%%%
%%%%%%%%%%%%%%%%%%%%%%%%%%%%%%%%%%%%%%%%%%%%%%%%%%%%%%%%%%%

%--------------------------------------------------------
% Problem 1
\begin{shaded}

\begin{problem}
\section{Problem 1: Optimal mechanism}
Consider an auction for one item with two bidders whose values are independently and uniformly drawn from [1, 3] and [2, 4], respectively.\\
(a) (7 points) Find the optimal auction in which truthful bidding is a dominant strategy that maximizes the expected revenue of the seller. You need to describe both the allocation rule and the payment rule. Explain your answer.\\
(b) (3 points) Find the expected revenue of the optimal auction you find in Part (a).Explain your answer.
\end{problem}

\begin{solution}
\textbf{Setup and virtual values.} Player 1's value $v_1 \sim U[1,3]$, Player 2's value $v_2 \sim U[2,4]$.  
For a uniform $U[a,b]$, the virtual value is $\varphi(v)=v-\frac{1-F(v)}{f(v)}=2v-b$. Hence
\[
\varphi_1(v_1)=2v_1-3,\qquad \varphi_2(v_2)=2v_2-4.
\]
Both $\varphi_i$ are strictly increasing (regular), so Myerson's optimal auction allocates to the bidder with the highest nonnegative virtual value and is implemented by a DSIC mechanism with a critical-value payment.

\medskip
\noindent\textbf{Reserve values.} The individual reserves $r_i$ solve $\varphi_i(r_i)=0$:
\[
r_1=\tfrac{3}{2}=1.5,\qquad r_2=2.
\]

\medskip
\noindent\textbf{(a) Allocation rule.} Allocate the item to maximize $\max\{0,\varphi_1(v_1),\varphi_2(v_2)\}$, i.e.,
\begin{itemize}
  \item If $v_1<1.5$ and $v_2<2$: allocate to no one.
  \item If $v_1\ge 1.5$ and $v_2<2$: allocate to Player 1.
  \item If $v_1<1.5$ and $v_2\ge 2$: allocate to Player 2.
  \item If $v_1\ge 1.5$ and $v_2\ge 2$: compare $\varphi_1$ and $\varphi_2$.
  Since $\varphi_1\ge \varphi_2 \iff 2v_1-3\ge 2v_2-4 \iff v_1\ge v_2-\tfrac{1}{2}$,
  \[
  \begin{cases}
  \text{allocate to Player 1} & \text{if } v_1\ge v_2-\tfrac{1}{2},\\
  \text{allocate to Player 2} & \text{if } v_1< v_2-\tfrac{1}{2}.
  \end{cases}
  \]
\end{itemize}

\noindent\textbf{(a) Payment rule (critical values).} The winner pays the smallest report that would still secure the allocation, holding the opponent's report fixed.
\begin{itemize}
  \item If Player 1 wins (this necessarily has $v_2\ge 2$), the threshold for $v_1$ is
  \[
  v_1^{\ast}=\max\{1.5,\,v_2-\tfrac{1}{2}\}=v_2-\tfrac{1}{2}\quad(\text{since }v_2\ge 2).
  \]
  Thus Player 1's payment is
  \[
  p_1(v_1,v_2)=v_2-\tfrac{1}{2}.
  \]
  \item If Player 2 wins:
  \[
  p_2(v_1,v_2)=
  \begin{cases}
  2, & \text{if } v_1<1.5\quad(\text{opponent below reserve}),\\[2mm]
  v_1+\tfrac{1}{2}, & \text{if } v_1\ge 1.5\quad(\text{threshold on the boundary } v_2=v_1+\tfrac{1}{2}).
  \end{cases}
  \]
\end{itemize}
By construction the allocation is monotone in each bidder's own report and payments are the corresponding critical values, so the mechanism is DSIC and IR.

\medskip
\noindent\textbf{(b) Expected revenue.} By Myerson's revenue equivalence,
\[
\mathbb{E}[\text{Revenue}]
=\mathbb{E}\big[\max\{0,\varphi_1(v_1),\varphi_2(v_2)\}\big].
\]
With joint density $(1/2)\cdot(1/2)=1/4$ on $[1,3]\times[2,4]$, split the domain into:
\[
\begin{aligned}
&\text{(i) } v_1\in[1,1.5),\ v_2\in[2,4]:\ \text{winner is 2, contributes } \varphi_2=2v_2-4;\\
&\text{(ii) } v_1\in[1.5,3],\ v_2\in[2,4]:\ \text{winner is } \max\{\varphi_1,\varphi_2\} \text{ across the line } v_2=v_1+\tfrac{1}{2}.
\end{aligned}
\]
A direct calculation gives
\[
\iint \max\{0,\varphi_1,\varphi_2\}\,dv_2\,dv_1
=\underbrace{2}_{\text{(i)}}+\underbrace{\frac{9}{4}+\frac{39}{8}}_{\text{(ii)}}=\frac{73}{8}.
\]
Multiplying by the joint density $1/4$ yields
\[
\boxed{\ \mathbb{E}[\text{Revenue}]=\frac{73}{32}=2.28125\ }.
\]

\end{solution}


\end{shaded}


\pagebreak


% Problem 2
\begin{shaded}

\begin{problem}
\section{Problem 2: Optimal mechanism}
Consider two people (“bidders”) who can be potentially matched. Their values of being matched are independently and uniformly drawn from [0, 1]. If they are not matched then both get 0 value. The match maker wants to implement a mechanism by soliciting bids from the bidders to determine whether they are matched and how much they pay. The bidder’s utility equals to her valuation of the outcome, minus her payment to the match maker.\\
(a) (7 points) Find the optimal mechanism in which truthful bidding is a dominant strategy that maximizes the expected revenue of the match maker. You need to describe both the allocation rule and the payment rule. Explain your answer.\\
(b) (3 points) Find the expected revenue of the optimal mechanism you find in Part (a). Explain your answer.
\end{problem}

\begin{solution}
\textbf{Virtual values and approach.}
Recall that for a uniform $U[a,b]$, the virtual value is
\[
\varphi(v)=v-\frac{1-F(v)}{f(v)}=2v-b.
\]
Here $v_i\sim U[0,1]$, so for each bidder $i=1,2$,
\[
\varphi_i(v_i)=2v_i-1.
\]
Both $\varphi_i$ are strictly increasing (regular), so we can relax IC and maximize the expected \emph{nonnegative} virtual surplus; DSIC is then implemented by a monotone allocation with critical-value payments (Myerson).

\medskip
\noindent\textbf{(a) Allocation rule.}
Match if and only if the sum of virtual values is nonnegative:
\[
\varphi_1(v_1)+\varphi_2(v_2)=2(v_1+v_2)-2\ \ge 0
\quad\Longleftrightarrow\quad
v_1+v_2\ \ge\ 1.
\]
Thus the allocation rule is:
\[
\text{Match the two bidders if } v_1+v_2\ge 1;\ \text{otherwise, do not match.}
\]

\medskip
\noindent\textbf{(a) Payment rule (critical values).}
Fix the opponent's report $v_{-i}$. Bidder $i$ is matched iff $v_i \ge 1-v_{-i}$.
Hence bidder $i$'s \emph{critical value} (the smallest report that still yields a match) is
\[
v_i^{\ast}=1-v_{-i}.
\]
Therefore the DSIC payment of bidder $i$ is her critical value whenever she is matched, and $0$ otherwise:
\[
p_i(v_i,v_{-i})=
\begin{cases}
1 - v_{-i}, & \text{if } v_i + v_{-i} \ge 1 \ (\text{matched}),\\
0, & \text{if } v_i + v_{-i} < 1 \ (\text{not matched}).
\end{cases}
\]
(Equivalently, when matched the total payment is $p_1+p_2=(1-v_2)+(1-v_1)=2-(v_1+v_2)$.)

\medskip
\noindent\textbf{IC/IR.}
For each bidder, the allocation is monotone in own report and the payment equals the corresponding critical value; thus truthful bidding is a dominant strategy (DSIC). If not matched, payment is $0$; if matched at threshold, utility is $v_i - (1 - v_{-i}) \ge 0$. Hence IR holds.

\medskip
\noindent\textbf{(b) Expected revenue.}
By Myerson's lemma, expected revenue equals the expected nonnegative virtual surplus:
\[
\mathbb{E}[\text{Rev}]
=\mathbb{E}\big[(\varphi_1(v_1)+\varphi_2(v_2))_+\big]
=\mathbb{E}\big[(2(v_1+v_2)-2)_+\big].
\]
Let $S=v_1+v_2$. For $v_1,v_2\sim U[0,1]$ i.i.d., $S$ has the Irwin--Hall density
\[
f_S(s)=
\begin{cases}
s, & 0\le s\le 1,\\
2-s, & 1\le s\le 2.
\end{cases}
\]
Therefore,
\[
\mathbb{E}[\text{Rev}]
=\int_{1}^{2} (2s-2)\,f_S(s)\,ds
=\int_{1}^{2} (2s-2)(2-s)\,ds
=\int_{1}^{2} (2-s)^2\,ds
=\boxed{\ \tfrac{1}{3}\ }.
\]
(Alternatively, integrating total payments $2-S$ over the matching region $\{S\ge 1\}$ gives the same value.)

\end{solution}

\end{shaded}

\end{document}