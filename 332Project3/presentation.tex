\documentclass{beamer}
\usetheme{Madrid}
\usecolortheme{default}


\usepackage[T1]{fontenc}
\usepackage[utf8]{inputenc}
\usepackage{amsmath,amssymb,bm,mathtools}
\usepackage{xcolor}
\usepackage{hyperref}
\usepackage{microtype}

\graphicspath{{figures/}}
\usepackage{booktabs}


\title[Project 3]{Project 3}
\subtitle{CS 332, Fall 2025}
\author{Northwestern University}
\date{5 November, 2025}

\begin{document}

\maketitle

\begin{frame}{Outline}
    \tableofcontents
\end{frame}

\AtBeginSection[]{
    \begin{frame}{Outline}
        \tableofcontents[currentsection]
    \end{frame}
}

\section{Outcomes from No-regret Learning in Games}

\begin{frame}{Setting: Repeated First-Price Auction}
    \begin{itemize}
        \item Full Information + Full Feedback
        \item Two bidders; each bidder's value \(v\) is fixed across rounds
        \item FPA; utility if win: \(u = v - b\); else 0
        \item Tie-breaking : If bids are even, allocate a thing randomly
    \end{itemize}
\end{frame}

\begin{frame}{Part 1 Setup}
  \begin{itemize}
    \item Rounds and runs: $n\approx 10000$, $n_{mc}=100$
    \item Discretization and values: $k=100$, $v_1=v_2=1.0$
    \vspace{0.8em}
    \item Algorithms compared
    \begin{itemize}
        \item \textbf{Myopic}: maximizes current-round $\mathbb{E}[u(b)] = (v-b)\,\Pr(\text{win}\mid b)$ using empirical opponent bids (full feedback), with tie probability 0.5.
        \item \textbf{Flexible}: exponential weights over $k$ discretized bids; full-feedback updates all arms each round; $\epsilon = \sqrt{\log k / n}$.
    \end{itemize}
  \end{itemize}
\end{frame}

\begin{frame}{Flexible (Exponential Weights)}
    \begin{itemize}
        \item Exponential weights over $k$ bids $b_j\in[0,v]$; full-feedback updates all arms each round.
        \item Utility per arm: $u_j=(v-b_j)\cdot \Pr(\text{win}_j)$ with $\Pr(\text{win}_j)\in\{0,0.5,1\}$ from last opponent bid.
        \item Cumulative payoffs $V_j \leftarrow V_j + u_j$; sample with
              $\pi_j \propto (1+\epsilon)^{V_j/h}$, $\ \epsilon=\sqrt{\log k/n}$.
        \item First round bids near $v$ on the grid; uses shared grid and tie tolerance.
    \end{itemize}
\end{frame}

\begin{frame}{Part 1 Results: Flexible(OPT lr) vs Flexible(OPT lr)}
  \begin{minipage}[t]{0.48\linewidth}
    \centering
    \includegraphics[width=\linewidth]{ew_vs_ew/bid_evolution.png}
    \vspace{0.3em}
    \small Bid
  \end{minipage}\hfill
  \begin{minipage}[t]{0.48\linewidth}
    \centering
    \includegraphics[width=\linewidth]{ew_vs_ew/regret.png}
    \vspace{0.3em}
    \small Regret
  \end{minipage}
    %%% TAKEAWAYS HERE %%%
    \begin{itemize}
        \item This is normal setting in repeated FPA.
        \item Players here have same values and same strategy (learning rate) 
    \end{itemize}
\end{frame}

\begin{frame}{Robustness}
  \begin{itemize}
    \item No pure NE in complete-info repeated FPA except for bidding strategy of b = v because both players have incentive to deviate from NE to bid higher as long as below value.
    \item Ran with different learning rate $\epsilon$, a new algorithm.
    \item Flexible maintains higher average utility and win rate across tested settings.
  \end{itemize}
\end{frame}

% Maybe we don't need these next two
\begin{frame}{Myopic (one-step expected utility)}
We also developed this algorithm independently from the Exponential Weights Algorithm. It is a reactive but not strategic adaptive strategy that performs remarkably well in the long run for the two-player case.
    \begin{itemize}
        \item Full-feedback: uses opponent bids from history.
        \item Discrete bid grid of size $k$ in $[0,v]$; first round bids $\approx 0.5v$.
        \item For each $b$ on the grid, compute $P(\text{win}\mid b)=\Pr(\text{opp}<b)+0.5\,\Pr(\text{opp}=b)$.
        \item Choose $b^\star=\arg\max_b (v-b)\,P(\text{win}\mid b)$; ties handled with 0.5.
    \end{itemize}
\end{frame}

\begin{frame}{Part 1 Results: Myopic vs Flexible(OPT lr)}
  \begin{minipage}[t]{0.48\linewidth}
    \centering
    \includegraphics[width=\linewidth]{332Project3/figures/empirical_vs_ew/bid_evolution.png}
    \vspace{0.3em}
    \small Bid
  \end{minipage}\hfill
  \begin{minipage}[t]{0.48\linewidth}
    \centering
    \includegraphics[width=\linewidth]{332Project3/figures/empirical_vs_ew/regret.png}
    \vspace{0.3em}
    \small Regret
  \end{minipage}
    %%% TAKEAWAYS HERE %%%
    \begin{itemize}
        \item Myopic achieves lower regret and higher win rate across MC runs
        \item Myopic bids are smoother over time; Flexible reacts to recent bids.
    \end{itemize}
\end{frame}

\begin{frame}{Part 1 Results: Flexible(3x OPT lr) vs Flexible(OPT lr)}
  \begin{minipage}[t]{0.48\linewidth}
    \centering
    \includegraphics[width=\linewidth]{optimal_lr_vs_3x_optimal_lr/bid_evolution.png}
    \vspace{0.3em}
    \small Bid
  \end{minipage}\hfill
  \begin{minipage}[t]{0.48\linewidth}
    \centering
    \includegraphics[width=\linewidth]{optimal_lr_vs_3x_optimal_lr/regret.png}
    \vspace{0.3em}
    \small Regret
  \end{minipage}
    %%% TAKEAWAYS HERE %%%
    \begin{itemize}
        \item Speed and stability form a trade-off — faster adaptation means less stability.
    \end{itemize}
\end{frame}

%%% Still wondering if this is a trivial solution. Note to self: evidence, robustness (just more rounds), maybe compare to diff algo.
%%% We are allowed full feedback of course - reading opponent bid each round - so the approach should be fine
\section{Manipulability of No-regret Learners in Games}
\begin{frame}{Part 2 Setup}
    \begin{itemize}
        \item Rounds and runs: $n=10000$, $n_{mc}=100$
        \item Discretization and values: $k=100$, $v_1=0.9,\ v_2=0.3$
        \item Exploitation observation rounds: $5$
        \vspace{.8em}
        \item Manipulation Setup:
        \begin{itemize}
            \item Opponent to be exploited will use Flexible algorithm
            \item \textbf{Our exploitation}:
            \begin{itemize}
                \item Observe for initial rounds (e.g., bid $0.2v$) and track opponent bids.
                \item Exponential smoothing predicts when opponent drops below a threshold.
                \item When predicted low, bid to win with positive margin (handle ties at 0.5).
            \end{itemize}
        \end{itemize}
    \end{itemize}
\end{frame}

\begin{frame}{Part 2 Results: Exploiting Fixed-Lowering Opponent}
  \begin{minipage}[t]{0.48\linewidth}
    \centering
    \includegraphics[width=\linewidth]{332Project3/figures/ew_vs_exploitation/bid_evolution.png}
    \vspace{0.3em}
    \small Bid
  \end{minipage}\hfill
  \begin{minipage}[t]{0.48\linewidth}
    \centering
    \includegraphics[width=\linewidth]{332Project3/figures/ew_vs_exploitation/regret.png}
    \vspace{0.3em}
    \small Regret
  \end{minipage}
    %%% TAKEAWAYS HERE %%%
    \begin{itemize}
        \item As shown in this figure, the bids fluctuate over time due to the opponent’s exploiting strategy.
    \end{itemize}
\end{frame}

\begin{frame}{Part 2 Results: Exploiting Fixed-Lowering Opponent(detail}
  \begin{minipage}[t]{0.48\linewidth}
    \centering
    \includegraphics[width=\linewidth]{332Project3/figures/ew_vs_exploitation_for_detail/bid_evolution.png}
    \vspace{0.3em}
    \small Bid
  \end{minipage}\hfill
  \begin{minipage}[t]{0.48\linewidth}
    \centering
    \includegraphics[width=\linewidth]{332Project3/figures/ew_vs_exploitation_for_detail/regret.png}
    \vspace{0.3em}
    \small Regret
  \end{minipage}
\end{frame}

\section{Supplementary}
\begin{frame}{Suppelementary : Battle of Sexes}
  \begin{itemize}
    \item Battle of Sexes:
    \begin{itemize}
      \item Two players, each with two actions(A,B) and payoff matrix:
      \begin{align*}
        \begin{pmatrix}
          2 & 0 \\
          0 & 4
        \end{pmatrix}
      \end{align*}
    \end{itemize}
    \item NE: $(A, A)$ and $(B, B)$ (observable)
    \item Mixed NE: $(0.33, 0.67)$ and $(0.67, 0.33)$ (unobservable)
  \end{itemize}
\end{frame}

\begin{frame}{Suppelemtary : Results}
  \textbf{Results:}
      \begin{center}
        \begin{minipage}[t]{0.48\linewidth}
          \centering
          \includegraphics[width=\linewidth]{332Project3/figures/flexible_vs_empirical_battle_of_sexes/flexible_vs_empirical_battle_of_sexes_final_ne_convergence.png}
        \small \textit{Flexible vs Empirical}
        \end{minipage}
        \hfill
        \begin{minipage}[t]{0.48\linewidth}
          \centering
          \includegraphics[width=\linewidth]{332Project3/figures/flexible_vs_empirical_battle_of_sexes/flexible_vs_empirical_battle_of_sexes_action_profile_frequencies.png}
        \small \textit{Flexible vs Empirical}
        \end{minipage}
      \end{center}
    \textbf{Takeaways:}
    \begin{itemize}
      \item Flexible and FTL can achieve the NE in the Battle of Sexes.
      \item Empirical does not achieve the NE in the Battle of Sexes.
    \end{itemize}
\end{frame}

\begin{frame}{Suppelemtary : Results}
  \textbf{Results:}
  \begin{center}
    \begin{minipage}[t]{0.48\linewidth}
      \centering
      \includegraphics[width=\linewidth]{332Project3/figures/flexible_vs_flexible_battle_of_sexes/flexible_vs_flexible_battle_of_sexes_final_ne_convergence.png}
      \\[4pt]
    \small \textit{Flexible vs Flexible}
    \end{minipage}
    \hfill
    \begin{minipage}[t]{0.48\linewidth}
      \centering
      \includegraphics[width=\linewidth]{332Project3/figures/flexible_vs_flexible_battle_of_sexes/flexible_vs_flexible_battle_of_sexes_action_profile_frequencies.png}
      \\[4pt]
    \small \textit{Flexible vs Flexible}
    \end{minipage}
  \end{center}
  \textbf{Takeaways:}
  \begin{itemize}
    \item Flexible and FTL can achieve the NE in the Battle of Sexes.
    \item Empirical does not achieve the NE in the Battle of Sexes.
  \end{itemize}
\end{frame}





\begin{frame}{Usage of AI}
    AI was used for figures, code, and design; final review and responsibility by the authors.
\end{frame}

\end{document}